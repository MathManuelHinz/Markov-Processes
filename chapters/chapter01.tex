\chapter{Continuous time Markov chains}

Two motivating examples:

%TODO

\section{Markov chain, transition function, infinitesimal description}

We have a \dhighlight{countable state space \(\cS\)} and define
\begin{itemize}
    \item \(\Omega=\{\text{right-continuous functions }\omega:\R_+\to\cS \text{ with finitely many jumps in any finite time intervals}\}\) 
    \item \(\sigma\)-algebra \(\cF\coloneqq \sigma(\{\omega\to\omega(t)\text{ which is measurable }\forall t\geq 0\})\)\marginnote{measurable w.r.t.?}
    \item time shift operator: \((\theta_s \omega)(t)\coloneqq \omega(s+t)\)
\end{itemize}

\begin{definition}\label{def:1.1}
    Denote \(X(t,\omega)\coloneqq \omega(t)\). 
    Assume we have a collection of probability measures \(\{\bP^x,x\in\cS\}\) on \((\Omega,\cF)\),
    a right-continuous filtration \(\{\cF_t,t\geq 0\}\) on \((\Omega,\cF)\). 
    \(X\) is a \dhighlight{continuous time Markov chain (MC)} if 
    \begin{enumerate}
        \item[(a)] \(X\) is adapted to \(\{\cF_t,t\geq 0\}\)
        \item[(b)] \dhighlight{Initial condition:} \(\bP^x(X(0)=x)=1\)
        \item[(c)] \dhighlight{Markov property:} \(\forall x\in \cS,Y \text{ measurable function on }\Omega,s\geq 0\):\[\bE^x\left(Y\circ \theta_s\mid \cF_s\right)=\bE^{X(s)}(Y), \bP^x \as\]   
    \end{enumerate}

\end{definition}

\begin{example*}
    \(Y=\max_{0\leq t\leq 1} X(t)\to Y\circ \theta_s = \max_{s\leq t\leq s+1} X(t)\)
\end{example*}

\dhighlight{Here:} Time homogenous processes! 

\begin{definition}\label{def:1.2}
    A \dhighlight{transition function} is function \(p_t(x,y),x,y\in\cS,t\geq 0\) s.t. 
    \begin{enumerate}
        \item[(a)] \dhighlight{Positivity:} \(p_t(x,y)\geq 0\)
        \item[(b)] \dhighlight{Normalized:} \(\sum_{y\in\cS} p_t(x,y)=1\)\marginnote{We will see while constructing, that this part is tricky, might just be \(\leq 1\)}
        \item[(c)] \dhighlight{Continuity:} \(\lim_{t\downarrow 0} p_t(x,x)=p_0(x,x)=1\)
        \item[(d)] \dhighlight{Chapman-Kolmogorov equation:} \(\forall s,t\geq 0\) \(\forall x,y\in \S\):\[p_{s+t}(x,y)=\sum_{z\in\cS} p_s(x,z)p_t(z,y)\] 
    \end{enumerate}
\end{definition}

Given a transition function, we can construct \(\bP^x\) as follows:

\dhighlight{Finite dimensional distributions}, \(0<t_1<\dots<t_n\): 
\begin{eqnarray*}
    \bP^x(X(t_1)=x_1,X(t_2=x_2),\dots,X(t_n)=x_n)\coloneqq p_{t_1}(x,x_1)\cdot p_{t_2-t_1}(x_1,x_2)\cdot\dots\cdot p_{t_n-t_{n-1}(x_{n-1},x_n)}
\end{eqnarray*}
extend to full time \(\R_+\) by the Kolmogorov(-Daniell) extension theorem, where the consistency,\(\bP^x(X(t)=y)=p_t(x,y)\), 
relations follow by the Chapman-Kolmogorov equation.

\begin{example*}
    \(0<t_2<\dots<t_n\)\marginnote{We are still in the same setting, therefore we use the same \(t_1\) as previously}
    \begin{equation*}
       \sum_{x_1\in\cS}\bP^x\left(X(t_1)=x_1,\dots,X(t_n)=x_n\right)=\bP^x\left(X(t_2)=x_2,\dots,X(t_n)=x_n\right)
    \end{equation*}
\end{example*}

By modeling we often think at the basic biological/ physical properties of the system \(\to\)
typically we have \textit{transition rates}, because for \( x\neq y: 
\begin{cases}
    p_\epsilon(x,y)=O(\epsilon),& p_\epsilon(x,x)=1-O(\epsilon)\\
      p_0(x,y)=0,&p_0(x,x)=1
\end{cases}
\)
\begin{definition}\label{def:1.3}
    For a Markov chain \(X\) we define the \dhighlight{transition rates} from \(x\) to \(y\) (\(x\neq y\)) by:
    \[\tilde{q}(x,y)\coloneqq \frac{d}{dt}p_t(x,y)\mid_{t=0}\]
\end{definition}

\begin{definition}
    A \dhighlight{\(Q\)-matrix (or generator)} is a collection of numbers \(\{q(x,y):x,y\in\cS\}\) s.t.:
    \begin{enumerate}
        \item[(a)] \(q(x,y)\geq 0\forall x\neq y\)
        \item[(b)] \(\sum_{y\in\cS} q(x,y)=0\), Notation: \(c(x)\coloneqq q(x,x)=\sum_{y\in\cS\setminus\{x\}} q(x,y)\)\marginnote{\(c(x)\) is the rate of leaving site \(x\)}   
    \end{enumerate}
\end{definition}

\begin{tcolorbox}[enhanced,breakable,
	title=Warning,colback=red!20!white]
    It is not automatic that transition rates gives a \(Q\)-matrix
    For instance: \(\tilde{q}(x,y)=\infty\)
\end{tcolorbox}

For \highlight{finite} state space,
    
Markov chain \(\iff\) Transition function \(\iff\) \(Q\)-matrix


\dhighlight{Goal:} Under which condition is the equivalence still true? 

\section{Examples}

\begin{example*}[Discrete to continuous MC]
    \dhighlight{Given:} Markov chain \(Y\) in discrete time \(t\in\Z_+\), i.e. transition matrix \(P=(P(x,y))_{x,y\in\cS}\)
    \[\bP(Y(n+1)=y\mid Y(n)=x)=P(x,y)\]
    Consider a Poisson process(PP) with intensity \(1\), at each event time of the PP, there will be a jump of the 
    continuous time MC \(X\). The jumps follow the discrete time MC \(Y\)

    \(\implies p_t(x,y)=\sum_{n\geq 0}\frac{e^{-t}t^n}{n!}P^n(x,y)\)

    in this case \(Q=P-1\)
\end{example*}

\begin{example*}[Finite $\cS$]
    \(p_t(x,y)=\left(e^{tQ}(x,y)\right)\coloneqq \sum_{n\geq 0}\frac{t^n}{n!}Q^n(x,y)\) \marginnote{Here will be one of the main problems: If \(Q\) is finite / a normed operator, everything is well defined. Otherwise we have to use a different definition of the exponetial \dots}
\end{example*}

\begin{example*}[Birth and death processes]
    \(\cS=\{0,1,\dots,\},X(t)=\text{Population size at time } t\). Then 
    \begin{itemize}
        \item \(q(k,k+1)=\rho_k, k\geq 0\)
        \item  \(q(k+1,k)=\lambda_k,g\geq 1\)
        \item \(q(k,k)=-\rho_k-\lambda_k\) with \(\lambda_0=0\)
        \item which implies \(q(k,l)=0\forall |k-l|\geq 2\)
    \end{itemize}
    Depending on the choice of \(\rho_k,\lambda_k\) it is possible that the chain goes to \(\infty\) in finite time.
\end{example*}

\beginlecture{02}{10.10.2024}
\section{From MC to \(Q\)-matrices}

\subsection{MC \(\to\) Transition functions}
\begin{theorem}\label{thm:1.5}
    Let \(X\) be a Markov chain. Then \(p_t(x,y)\coloneqq\bP^x(X(t)=y)\).
    \begin{enumerate}
        \item[(a)] \(p_t(x,y)\) is a transition function 
        \item[(b)]\(p_t(x,y)\) determines uniquely \(\bP^x\)  
    \end{enumerate}
\end{theorem}

\begin{proof}
    \dhighlight{(a):} Positivity and normalization follows by the properties of \(\bP^x\).
    \(\tau\coloneqq \inf\{t\in\R_+\mid X(t)\neq X(0)\}\) is a.s. positive. \(p_t(x,x)\geq \bP^x(\tau>t)\), which 
    implies \(1=p_0(x,x)\geq \lim_{t\downarrow 0}p_t(x,x)\geq \lim_{t\downarrow 0}\bP^x(\tau >t)=1\). 
    
    We still need to verify the Chapman-Kolmogorov equation:
    Define \(Y=1_{X(t)=y}\implies Y\circ \theta_s=1_{X(t+s)=y}\) 
    \begin{align*}
        p_{s+t}(x,y) &=\bP^x(X(s+t)=y)=\bE^x\left(1_{X(s+t)=y}\right)\\
        &=\bE^x\left(\bE^x(1_{X(s+t)=y}\mid \cF_s)\right)\\
        &\stackrel{\text{MP}}{=}\bE^x\left(\bE^{X(s)\left(1_{X(t)=y}\right)}\right)\\
        &=\sum_{z\in S}p_s(x,z)\underbrace{\bE^z\left(1_{X(t)=y}\right)}_{=p_t(z,y)}
    \end{align*}
    \dhighlight{(b):} By the Markov property \(\forall 0<t_1<\dots<t_n,x_1,\dots,x_n\in S\)
    \begin{align*}
        \bP^x\left(X(t_1)=x_1,\dots X(t_n)=x_n\right)=p_{t_1}(x_1,x_1)\cdot p_{t_2-t_1}(x_1,x_2)\cdot \dots\cdot p_{t_n-t_{n-1}}(x_{n-1},x_{n})
    \end{align*}
    using the Kolmogorov extension theorem  \(\bP^x\) is uniquely determined by \(p_t\).
\end{proof}

\subsection{Transition functions to \(Q\)-matrices}

\begin{proposition}\label{prop:1.6}
    Let \(p_t(x,y)\) be a transition function.
    \begin{enumerate}
        \item[(a)] \(\forall t\geq 0,x\in S,p_t(x,x)>0\)
        \item[(b)] If \(p_t(x,x)=1\) for some \(t>0\), then \(p_s(x,x)=1\forall s\geq 0\)
        \item[(c)] \(\forall x,y\in S, p_t(x,y)\) is uniformly continuous in \(t\), due to 
        \[\vert p_{t+s}(x,y)-p_t(x,y)\vert\leq 1-p_{s}(x,x)\]
    \end{enumerate}
\end{proposition}

\begin{remark}
    \begin{enumerate}
        \item[(a)] differs from the discrete time markov chain \(\to\) no issues of periodicity 
        \item[(b)] The M.C. if it reaches site \(x\), stays forever in \(x\). \(x\) is an \dhighlight{absorbing state}
    \end{enumerate}
\end{remark}

\begin{proof}
    \dhighlight{(a):} Since \(\lim_{x\downarrow 0}p_t(x,x)=1\implies p_s(x,x)>0\) for \(s \in [0,t]\) for some small \(t\).
    Therefore \(\forall s\in[0,t]:p_{t+s}(x,x)\stackrel{\text{C.Kolmogorov}}{\geq}p_t(x,x)p_s(x,x)>0\), which we can iterate to get the claim. 
   
    \dhighlight{(b):} \(p_{s+t}(x,x)=\sum_{z\in S} p_s(x,z)p_t(z,x)\leq p_s(x,x)p_t(x,x)+\underbrace{\sum_{z\neq x}p_s(x,z)}_{=1-p_s(x,x)}\cdot 1\)\marginnote{still good, since \(p_s(x,z)\) will be \(0\)}
    Then \begin{align*}
        p_{s+t}(x,x)\leq 1-\underbrace{p_s(x,x)}_{\stackrel{(a)}{\geq 0}}(1-p_t(x,x)).
    \end{align*}
    If \(p_{s+t}(x,x)=1\implies p_t(x,x)=1\). \(\forall v\in [t+s,2(t+s)]: p_v(x,x)\geq p_{\underbrace{v-(t+s)}_{\in [0,t+s]}}(x,x)p_{t+s}(x,x)=1\cdot 1=1\)
    
    \dhighlight{(c):} \begin{align*}
        p_{s+t}(x,y)-p_t(x,y)&=\sum_{z\in S}p_s(x,z)p_t(s,y)-p_t(x,y)\\
        &=\underbrace{(p_s(x,x)-1)p_t(x,y)}_{\leq 0}+\underbrace{\sum_{z\neq x}p_s(x,z)p_t(z,y)}_{\geq 0}
    \end{align*}
    Since \[|p_s(x,x)-1|p_t(x,y)\leq 1-p_s(x,x)\] and\[\sum_{z\neq x}p_s(x,z)\underbrace{p_t(z,y)}_{\leq 1}\leq 1-p_s(x,x)\]
    \[\implies |p_{s+t}(x,y)-p_t(x,y)|\leq 1-p_s(x,y)\]
\end{proof}

\begin{theorem}\label{thm:1.8}
    Let \(p_t(x,y)\) be a transition function. \begin{enumerate}
        \item[(a)]\(\forall x\in S\) the right derivative \[\tilde{c}(x)=-\tilde{q}(x,x)=-\frac{d}{dt}p_t(x,x)\mid_{t=0}\in[0,\infty]\]
        exists and \(p_t(x,x)\geq e^{-\tilde{c}(x)t},\forall t>0\)
        \item[(b)] I \(tilde{c}<\infty\implies \forall y\neq x\), the right-derivative
            \[\tilde{q}(x,y)\coloneqq \frac{d}{dt}p_t(x,y)\mid_{t=0}\in[0,\infty]\]
            exists and \(\sum_{y\in S}\tilde{q}(x,y)\leq 0\) 
        \item[(c)] If for some  \(x\in S,\tilde{c}(x)<\infty\) and \(\sum_{y\in S} \tilde{q}(x,y)=0\), then \(p_t(x,y)\) is \(C^1\) in time for this \(x\) and \(y\in S\).
            Moreover, it satisfies the \dhighlight{Kolmogorov- backwards equation}: 
            \begin{equation}\label{eq:kolmogorov_backwards}\frac{d}{dt}p_t(x,y)=\sum_{z\in S} \tilde{q}(x,z)p_t(z,y)\end{equation} 
    \end{enumerate}
\end{theorem}

\begin{proof}
    \dhighlight{(a)} Let \(f(t)\coloneqq -\ln(p_t(x,x))\geq 0\). By \(p_{s+t}(x,x)\geq p_s(x,x)p_t(x,x)\) it follows 
    \begin{align*}
        f(t+s)\leq f(s)+f(t),
    \end{align*}
    the function \(f\) is subadditive, which implies \marginnote{In particular, the limit exists}\[\lim_{t\downarrow 0} \frac{f(t)}{t}=\sup_{t>0}\frac{f(t)}{t}\in[0,\infty]\]
    \begin{align*}
        \implies \lim_{t \downarrow 0}\frac{f(t)}{t}&=\frac{d}{dt}f(t)\mid_{t=0}=\frac{-\frac{d}{dt}p_t(x,x)}{\underbrace{p_t(x,x)}_{\to 1}}\mid_{t=0}=\tilde{c}(x)\\
        \implies & \tilde{c}(x)\geq \frac{f(t)}{t}=-\frac{\ln p_t(x,x)}{t}\to p_t(x,x)\geq e^{-\tilde{c}(x)t}
    \end{align*}

    \dhighlight{(b)} by (a): \(1-p_t(x,x)\leq 1-e^{-\tilde{c}(x)t}\leq \tilde{c}(x)t\), which implies 
    \begin{align*}
        \sum_{y\neq x} p_t(x,y)\leq\tilde{c}(x)t.
    \end{align*}
    \begin{align*}
    \implies \tilde{q}(x,y)\coloneqq \limsup_{t\downarrow 0} \frac{p_t(x,y)}{t}\in[0,\underbrace{\tilde{c}(x)}_{<\infty}]
    \end{align*}
    Let \(\delta>0,n\in\N\):
    Let \(p_\delta=\{p_\delta(x,y)\}_{x,y\in S}\) be a discrete time chain.
    \begin{align*}
        p_{n\delta}(x,y)&\geq \sum_{k=0}^{n-1}\underbrace{p_{\delta}(x,x)^k}_{\geq p_\delta(x,x)^n\geq e^{-\tilde{c}(x)n\delta}}p_\delta(x,y)p_{(n-k-1)\delta}(y,y)\\
        \implies&\frac{p_{n\delta}(x,y)}{n\delta}\geq \frac{p_{\delta(x,y)}}{\delta}e^{-\tilde{c}(x)n\delta}\cdot \inf_{0\leq s\leq n\delta}p_s(y,y)
    \end{align*}

    Take a subsequence of \(\delta\downarrow 0\) such that \(n\delta\to t\) and 
    \(\lim_{\delta\downarrow 0 }\frac{p_{\delta}(x,y)}{\delta}=\tilde{q}(x,y)\)
    \begin{align*}
        \implies \liminf_{t\downarrow 0}\frac{p_t(x,y)}{t}\geq \tilde{q}(x,y)\cdot 1 \cdot 1
    \end{align*}
    which implies that the limit exists.
    \begin{align*}
        \sum_{y\neq x}\frac{p_t(x,y)}{t}&\leq \tilde{c}(x)\\
        \tilde{c}(x)&\geq \liminf_{t\downarrow 0}\sum_{y\neq x}\frac{p_t(x,y)}{t}\stackrel{\text{Fatou}}{\geq}\sum_{y\neq x}\liminf_{t\downarrow 0}\frac{p_t(x,y)}{t}=\sum_{y\neq x}\tilde{q}(x,y)\\
        &\implies \sum_y \tilde{q}(x,y)\leq 0
    \end{align*}

    \dhighlight{(c)} By Chapman-Kolmogorov:
    \begin{align*}
        \frac{p_{t+\epsilon}(x,y)-p_t(x,y)}{\epsilon}-\sum_{z}\tilde{q}(x,z)p_t(z,y)\\
        =\sum_{z\in S}\underbrace{\left(\frac{p_\epsilon(x,z)-\overbrace{p_0(x,z)}^{\delta_{x,z}}}{\epsilon}-\tilde{q}(x,z)\right)}_{=(\star)}p_t(z,y)
    \end{align*}
    For all \(z:\lim_{\epsilon\to 0} (\star)=0\)

    Take \(T\subset S\), \(|T|<\infty\implies \lim_{\epsilon\downarrow 0}\sum_{z\in T}(\star)\cdot p_t(z,y)=0\).
    Let \(x\in T\). \begin{align*}&\sum_{z\notin T}\left\vert\frac{p_{\epsilon}(x,z)}{\epsilon}-\tilde{q}(x,z) \right\vert \underbrace{p_(z,y)}_{\leq 1}\\
        &\leq \sum_{z\notin T} \frac{p_\epsilon(x,z)}{\epsilon}+\sum_{z\notin T}\tilde{q}(x,z)\stackrel{\sum \tilde{q(x,z)=0}}{=}\frac{1}{\epsilon}\left(1-\sum_{z\in T}p_\epsilon(x,z)\right)-\sum_{z\in T}\tilde{q}(x,z) 
    \end{align*}
    But \(\sum_x\tilde{q}(x,z)=0\stackrel{\epsilon\downarrow 0}{\to} - 2\sum_{z\in T}\tilde{q}(x,z)\stackrel{T\uparrow S}{\to}0\)
\end{proof}

\beginlecture{03}{15.10.2024}

\begin{remark}
    For M.C. also the \dhighlight{strong Markov property} holds:

    Let \(\tau\) be a stopping time and \(Y:\R_+\times \Omega\mapsto Y_t(\omega)\footnote{\text{shorthand notation ...}}\) measurable, then
    \[\bE^x(Y\circ\theta_{\tau}\mid \cF_t)=\bE^x(Y_\tau), \bP-\as\]
    on the set \(\{\tau<\infty\}\).
\end{remark}

\begin{align*}
    \frac{d}{dt}p_t(x,y)=\lim_{\epsilon\downarrow 0} \frac{p_{t+\epsilon}(x,y)-p_t(x,y)}{\epsilon}
\end{align*}
\marginnote{They are almost always equivalent. In computation one often uses the forward equation, in construction the backwards equation is preffered}
using \(\sum_{z}p_t(x,z)p_\epsilon(z,y)\) instead of \(\sum_{z}p_\epsilon(x,z)p_t(z,y)\)
we get the \dhighlight{Kolmogorov Forward Equation}\begin{equation}\label{eq:kolmogorov_forward}
    \frac{d}{dt}p_t(x,y)=\sum_{z\in\cS}p_t(x,z)q(z,y)
\end{equation}

\section{From \(Q\)-matrix to the Markov Chains}

Let \(Q=(q(x,y))_{x,y\in\cS}\) be a \(Q\)-matrix.

\subsection{The backwards equation}

\begin{proposition}\label{prop:1.12}
    Let \(p_t(x,y)\) be a uniformly bounded function of \(x,y,t\), then (a) is equivalent to (b):
    \begin{enumerate}
        \item[(a)]\(p_t(x,y)\) is \(C^1\) in \(t\), satisfies the KBE\footnote{Kolmogorov Backwards Equation}, with initial condition \(p_0(x,y)=\delta_{x,y}\)
        \item[(b)] \(p_t(x,y)\) is \(C^0\) in \(t\), and satisfies:\begin{equation}\label{eq:A}p_t(x,y)=\delta_{x,y}e^{-c(x)t}+\int_0^t ds e^{-c(x)(t-s)}\sum_{z\in \cS\setminus\{x\}}q(x,z)p_s(z,y)\end{equation}
    \end{enumerate}
\end{proposition}

\begin{proof}
    (a) \(\implies\) (b): KBE \(\iff\)
    \begin{align*}
        \frac{d}{dt}p_t(x,y)&=-c(x)p_t(x,y)+\sum_{z\in S\setminus\{x\}} q(x,z)p_t(z,y)\\
        \implies\frac{d}{dt}\left(e^{c(x)t}p_t(x,y)\right)&=c(x)e^{c(x)t}p_t(x,y)+e^{c(x)t}\left(-c(x)p_t(x,y)+\sum_{z\neq x}q(x,z)p_t(z,y)\right)\\
        &=e^{c(x)t}\sum_{z\neq x}q(x,z)p_t(z,y)\\
        &\stackrel{\int_0^t}{=} e^{c(x)t}p_t(x,y)-p_0(x,y)=\int_0^te^{c(x)s}\sum_{z\neq x}q(x,z)p_s(z,y)
    \end{align*}

    (b) \(\implies\) (a): The RHS of the equation is \(C^1\) in \(t\). The initial condition is also satisfied. Taking the derivate yields
    the \(\frac{d}{dt}p_t(x,y)=-c(x)p_t(x,y)+\sum_{z\in S\setminus\{x\}} q(x,z)p_t(z,y)\) which is equivalent to the KBE.
\end{proof}

We use proposition \ref{prop:1.12} to show existence of positive solution of the KBE. Idea: Use a \highlight{fixed point argument}:
\begin{itemize}
    \item \(p_t^{(0)}(x,y)=0\forall x,y\in\cS,t\geq 0\)
    \item For \(n\geq 0: p_t^{(n+1)}(x,y)=\delta_{x,y}e^{c(x)t}+\int_{0}^t ds e^{-c(x)(t-s)}\sum_{z\neq x}q(x,z)p_s^{(n)}(z,y)\)
\end{itemize}

\begin{lemma}\label{lem:1.13}
    \(\forall n\geq 0\): \begin{enumerate}
        \item[(a)] \(p_t^{(n)}(x,y)\geq 0\forall x,y\in\cS\)
        \item[(b)] \(\sum_{y\in\cS}p_t^{(n)}(x,y)\leq 1\forall x\in\cS,t\geq 0\)
        \item[(c)] \(p_t^{(n+1)}(x,y)-p_t^{(n)}(x,y)\geq 0\forall x,y\in\cS,t\geq 0\)   
    \end{enumerate}
\end{lemma}


\begin{proof}
    By induction. \(n=0\) is clearly satisfied for (a),(b),(c).

    (a) is obvious, since RHS are all positive terms.

    (b) \begin{align*}
        \sum_{y\in\cS} p_t^{(n+1)}(x,y)&=e^{-c(x)t}+\int_{0}^t ds e^{-c(x)(t-s)}\sum_{z\neq x}q(x,z)\underbrace{\sum_{y\in\cS}p_s^{(n)}(z,y)}_{\leq 1}\\
        &\leq e^{-c(x)t}+\int_{0}^t ds e^{-c(x)(t-s)} \underbrace{\sum_{z\neq x} q(x,z)}_{=c(x)}\\
        &=e^{-c(x)t}+ e^{-c(x)t}c(x)\int_0^t ds e^{c(x)s}=1
    \end{align*}
    (c) 
    \begin{align*}
        p_t^{(n+2)}(x,y)-p_t^{(n+1)}(x,y)&=\int_0^t e^{-c(x)(t-s)}\sum_{z\neq x} q(x,z)\underbrace{\left(p_s^{(n+1)}(z,y)-p_s^{(n)}(z,y)\right)}_{\geq 0}\geq 0\qedhere
    \end{align*}
\end{proof}

\(\implies \forall x,y,t\) we have \(0\leq p_t^{0}(x,y)\leq p_t^{1}(x,y)\leq \dots\leq 1\).

\(\implies\) there exists a limit, which we denote by \(p_t^*(x,y)\)
\begin{definition}\label{def:1.14}
    The limit \(p_t^{*}(x,y)\coloneqq \lim_{n\to\infty} p_t^{n}(x,y)\)
    is called the \dhighlight{minimal solution of the KBE}.
\end{definition}

\dhighlight{Question:} Is \(p^*\) a transition function?

\begin{theorem}\label{thm:1.15}
    \(p^*\) satisfies:
    \begin{enumerate}
        \item[(a)] \(p_t^*(x,y)\geq 0\) 
        \item[(b)] \(\sum_{y\in S}p_t^*(x,y)\leq 1\)
        \item[(c)] The Chapman-Kolmogorov equation 
        \item[(d)] satisfies (\ref{eq:A}) and thus the KBE
    \end{enumerate}
\end{theorem}

\begin{proof}
    (a) and (b) follow directly from lemma \ref{lem:1.13}.

    (d): Take the limits of \(p_t^{(n)}(x,y)\) and use monotone convergence.

    (c): Define \(\Delta_t^{(n)}\coloneqq p_t^{(n+1)}(x,y)-p_t^{(n)}(x,y)\geq 0\).

    Notice \(p_t^*(x,y)=\sum_{n\geq 0} \Delta_t^{(n)}(x,y)\) In lemma \ref{lem:1.17}:
     \(Delta_{t+s}^{(n)}(x,y)=\sum_{z\in\cS}\sum_{k=0}^n\Delta_s^{(k)}(x,z)\Delta_t^{(n-k)}(z,y)\)
     which then implies 
    \begin{align*}
        p_{t+s}^*(x,y)&=\sum_{n\geq 0}\Delta_{t+s}^{(n)}(x,y)=\sum_{z\in\cS}\sum_{n\geq 0}\sum_{k=0}^{n}\Delta_s^{(k)}(x,z)\Delta_t^{(n-k)}(z,y)\\
        &=\sum_{z\in\cS}\underbrace{\sum_{k\geq 0}\Delta_s^{(k)}(x,z)}_{p_s^*(x,z)}\underbrace{\sum_{n=k}^\infty \Delta_{t}^{(n-k)}(z,y)}_{p_t^*(z,y)}\\
        &=\sum_{z\in\cS} p_s^*(x,z)p_t^*(z,y)
    \end{align*}
    using monotone convergence.
\end{proof}

\begin{lemma}\label{lem:1.17}
    \(\Delta_{t+s}^{(n)}(x,y)=\sum_{z\in\cS}\sum_{k=0}^n\Delta_s^{(k)}(x,z)\Delta_t^{(n-k)}(z,y)\)
\end{lemma}

\begin{proof}
    Consider the Laplace transform of the equation w.r.t. \(s,t\), since the functions are all positive!
    %Tofix
    \[\int_0^\infty ds\int_0^\infty dt e^{-\lambda s}e^{-\mu t\Delta_{s+t}^{(n)}}\stackrel{?}{=}\sum_{z\in\cS}\sum_{k=0}^n\underbrace{\left[\int_0^\infty ds e^{-\lambda s}\Delta_s^{(k)}(x,z)\right]}_{\Psi_{k,\lambda}(x,z)}\underbrace{\left[\int_0^t e^{-\mu t}\Delta_{t}^{(n-k)}(z,y)\right]}_{\Psi_{n-k,\mu}(z,y)}\]
    Define \(\int_0^t ds e^{-\lambda s}\Delta_s^{(n)}(x,y)=\Psi_{n,\lambda}(x,y)\).
    Then for the RHS:\begin{align*}
        &=\sum_{z\in \cS}\sum_{k=0}^n\Psi_{k,\lambda}(x,z)\Psi_{n-k}\mu(z,y)
    \end{align*}
    For the LHS:
    \begin{align*}
       \int_0^\infty ds ds \int_0^\infty dt e^{-\lambda (s+t)}e^{-t(\mu-\lambda)}\Delta_{\underbrace{s+t}_{\eqqcolon u}}^{(n)}(x,y)&=\int_0^\infty ds \int_s^\infty du e^{-\lambda u}e^{-\mu(u-s)}\Delta_{u}^{(n)}(x,y)\\
        &=\int_0^\infty du \int_u^\infty ds e^{-(\lambda-\mu)s}e^{-\mu u}\Delta_{u}^{(n)}(x,y)=\frac{\Psi_{n,\mu}(x,y)-\Psi_{n,\lambda}(x,y)}{\lambda-\mu}\\
        &\stackrel{?}{=}\sum_{k=0}^n\sum_{z\in\cS}\Psi_{k,\lambda}(x,y)\Psi_{n-k,\mu}(z,y)
    \end{align*}
    \marginnote{Above the second equation is wrong, but the skip from the RHS to the 3rd line is correct}
    Another identity: \(\Psi_{n+1,\lambda}(x,y)=\sum_{z\neq x}\frac{q(x,z)}{\lambda + c(x)}\Psi_{n,\lambda}(z,y)\)

    Define the matrix \(A_\lambda(x,y)\coloneqq \frac{q(x,y)}{\lambda + c(x)}1_{z\neq x}\), then
    \begin{align*}
        \Psi_{n,\lambda}=(A_\lambda)^n\Psi_{0,\lambda}
    \end{align*}
    where \(\Psi_{0,\lambda}(x,y)=\int_0^\infty ds e^{-\lambda s}p_s^{(1)}(x,y)\) 
    and \(p_s^{(1)}(x,y)=\delta_{x,y}e^{-c(x)s}\).
    Then \( \Psi_{0,\lambda}(x,y)=\delta_{x,y}\frac{1}{\lambda + c(x)}\).

    \(\implies\)
    \marginnote{Reminder: For matrices \(X^n-Y^n=(X-Y)X^{n-1}+ Y(X-Y)X^{n-1}+\dots+Y^{n-1}(X-Y)\), since they don't commute}
    \begin{align*}
        \frac{\Psi_{n,\mu}-\Psi_{n,\lambda}}{\lambda-\mu} &= \frac{(A_\mu)^n\Psi_{0,\mu}-(A_\lambda)^n \Psi_{0,\lambda}}{\lambda-\mu}\\
        &=\frac{(A_\lambda)^n (\Psi_{0,\mu}-\Psi_{0,\lambda})}{\lambda-\mu}\sum_{k=0}^{n-1} (A_\lambda)^k\frac{A_\mu-A_\lambda}{\lambda-\mu}A^{n-k-1}\Psi_{0,\mu}
    \end{align*}
    
    \begin{align*}
        \frac{\Psi_{0,\mu}-\Psi_0,\lambda}{\lambda-\mu}(x,y)=\frac{\delta_{x,y}}{\lambda-\mu}\left(\frac{1}{c(x)+\mu}-\frac{1}{c(x)+\lambda}\right)=\Psi_{0,\mu}(x,y)\cdot \Psi_{0,\lambda}(x,y).
    \end{align*}
    Similarly
    \begin{align*}
        \frac{A_\mu-A_\lambda}{\lambda-\mu}(x,y)=\left(\Psi_{0,\lambda}A_\mu\right)(x,y)
    \end{align*}
    Then 
    \begin{align*}
        \frac{(A_\lambda)^n (\Psi_{0,\mu}-\Psi_{0,\lambda})}{\lambda-\mu}\sum_{k=0}^{n-1} (A_\lambda)^k\frac{A_\mu-A_\lambda}{\lambda-\mu}A^{n-k-1}\Psi_{0,\mu} &= [(A_\lambda)^n\Psi_{0,\mu}](x,y)+\sum_{k=0}^{n-1}\sum_{z}\underbrace{\left[A_\lambda^k\Psi_{0,\lambda}\right]}_{=\Psi_{k,\lambda}(x,z)}(x,z)\underbrace{\left(A_\mu^{n-k}\Psi_{0,\mu}\right)(z,y)}_{=\Psi_{n-k,\mu}(z,y)}\\
        &=\sum_{k=0}^n \sum_{\in\cS} \Psi_{k,\lambda}(x,z)\Psi_{n-k,\mu(z,y)}\qedhere
    \end{align*}

\end{proof}

\beginlecture{04}{17.10.2024}

\begin{theorem}\label{thm:1.19}
    \begin{enumerate}
        \item[(a)] If \(p_t(x,y)\) is a non-negative solution of the KBE \ref{eq:kolmogorov_backwards} with 
                   \(p_0(x,y)=\delta_{x,y}\implies\) \(p_t(x,y)\geq p_t^*(x,y)\)
        \item[(b)] If \(\sum_{y\in \cS}p_t^*(x,y)=1,\forall x\in\cS\implies p_t^*(x,y)\) is the unique transition function satisfying the KBE    
    \end{enumerate}
\end{theorem}

\begin{proof}
    \dhighlight{(a):} By induction: \(p_t(x,y)\geq p_t^{(n)}(x,y)\forall n\).
    
    For \(n=0\) \(p_t(x,y)\geq 0 =p_t^{(0)}(x,y)\).

    \begin{align*}
        p_t^{(n+1)}(x,y)&=\delta_{x,y}e^{-c(x)t}+\int_0^tds e^{-c(x)(t-s)}\sum_{z\neq x}q(x,z)\underbrace{p_s^{(n)}(x,y)}_{\leq p_s(x,y)}\\
        &\stackrel{p_t \text{satisfies KBE}}{\leq} p_t(x,y)
    \end{align*}

    \dhighlight{(b)}: \begin{align*}
        &=\sum_{y\in \cS}\underbrace{\left(p_t^*(x,y)-p_t^*(x,y)\right)}_{\geq 0}\geq 0\\
        &=\underbrace{\sum_{y\in\cS}p_t(x,y)}_{=1}-\underbrace{\sum_{y\in\cS} p_t^*(x,y)}_{=1}\\
        &\implies 0 \geq \sum_{y\in \cS}\left(p_t^*(x,y)-p_t^*(x,y)\right)\geq 0\\
        &\implies p_t(x,y)=p_t^*(x,y)\qedhere
    \end{align*}
\end{proof}

\begin{example}
    \(\cS=\{0,1,\dots,\},q(i,j)=\begin{cases}
        \beta_i & j=i+1\\
        -\beta_i & j=i\\
        0 & \text{otherwise}
    \end{cases}\) for some positive \(\beta_i\).
    One shows \(\sum_{j\in\cS} p_t^*(i,j)=1\forall t,i\in S\iff \sum_{k\geq 0}\frac{1}{\beta_k}<\infty\).

    What happens if \(\sum_{k\geq 0}\frac{1}{\beta_i}=\infty\)? Consider \(N(t)=\#\) of jumps before time \(t\),then 
    \(\bP(n(t)=\infty)>0\).
\end{example}

\subsection{The probabilistic construction}

\begin{lemma}\label{lem:1.21}
    Let \(X(t)\) be a continuous time M.C. and \(\tau\coloneqq \inf\{t:X(t)\neq X(0)\}\). Then there exists \(c(x)=[0,\infty]\) s.t. 
    \(\bP(\tau>t)=e^{-{c(x)t}}\)
\end{lemma}

\begin{proof}
    Use Markov property with \(Y=1_{X(r)=x,0\leq r\leq t-s}\):
    \begin{align*}
        \bP^x\left(X(r)=x,s\leq r\leq t\mid \cF_s\right)&=\bP^{X(s)}(\tau>t-s)=(\star)\\
        \bE^x(1_{\tau>s}\cdot \bP^x(X(r)=x,s\leq r \leq t\mid \cF_s))&=\bP^x(\tau>t)\\
        \bE^x(1_{\tau>s}\cdot \bP^x(X(r)=x,s\leq r \leq t\mid \cF_s))&\stackrel{(\star)}{=}\bE^x\left(1_{\tau>s}\bP^{X(s)}(\tau>t-s)\right)\\
        &=\bP^x(\tau>s)\bP^x(\tau>t-s)\\
        &\implies \bP^x(\tau>t)=\bP^x(\tau>s)\bP(\tau>t-s)
    \end{align*}
    Either \(\bP^x(\tau>t)=0\forall t\implies c(x)=\infty\) (jump before \(t\)) or \(\bP^x(\tau>t)>0\forall t\).
    Define \(f(t)\coloneqq -\ln\bP^x(\tau >t)\) with \(f(0)=0\) \(\implies f(t) \geq 0\).
    and  \(f(t)=f(s)+f(t-s)\implies f(t)=k\cdot t\).
\end{proof}

\dhighlight{Idea:} Divide the waiting times (which are exponetially distributed \(\sim \exp(c(x)),\exp(c(y)),\dots\)) and the \dhighlight{jumps}.
The jumps are given by a discrete time M.C..  

\dhighlight{(1)} Discrete time Markov chain \(\{Z_n,n\geq 0\}\) with transition matrix:
\[P=(p(x,y))_{x,y\in\cS}\]
If \(c(x)=0\):\[ p(x,y)=\begin{cases}
   1&x=y\\ 
    0&x\neq y
\end{cases}\]

If \(c(x)>0\):
\[p(x,y)\begin{cases}
    \frac{q(x,y)}{c(x)} & x\neq y\\
    0 & x=y
\end{cases}.\]

\dhighlight{(2)} Initial distribution: \(\pi=(\pi(x))_{x\in\cS}\) is a probability measure.

\dhighlight{(3)} Family of \dhighlight{independent} exponetially distributed waiting timesindependent \(\tau_{x,i}\sim\exp(c(x)),x\in\cS,i\geq 1.\)

If \(c(x)=0\to \tau_{x,i}=\infty\).

The joint distribution of \(Z_n\) are given by 
\[\bP(Z_0=x_0,\dots,Z_n=x_n)=\pi(x_0)p(x_0,x_1)\cdot\dots\cdot p(x_{n-1},x_n)\]
\remark{Careful, the waiting times are only independent given the positions of \(Z_n\)}
\begin{align*}
    \bP(Z_0=x_0,Z_1=x_1,\dots,Z_n=x_n,\tau_0>t_0,\tau_1>t_1,\dots,\tau_n>t_n) & =\pi(x_0)p(x_0,x_1)\cdot \dots\cdot p(x_{n-1},x_n)e^{-c(x_0)t_0}\cdot\dots\cdot e^(-c(x_n)t_n)
\end{align*}
where \(tau_k\) is the waiting time of the \((k-1)\)th jump.\marginnote{This should be \(k+1\), otherwise what does \(\tau_{-1}\) mean?}

Denote by \(N(t)\coloneqq \begin{cases}
    \min(n\geq 0)\mid \tau_0+\tau_1+\dots+\tau_n>t & \sum_{k\geq 0}\tau_k>t \\
    \infty & \text{otherwise}
\end{cases}\)

\(N(0)=0\). 
\begin{definition}\label{def:1.22}
    We define the continuous time Markov chain \(X(t)\) by 
    \(X(t)=Z_{N(t)}\) on \(\{N(t)<\infty\}\).
\end{definition}

\begin{proposition}\label{prop:1.23}
    \begin{enumerate}
        \item[(a)] \(p_t^{(n)}(x,y)=\bP(X(t)=y,N(t)<n\mid X(0)=x)\)
        \item[(b)] \(p_t^*(x,y)=\bP(X(t)=y,N(t)<\infty\mid X(0)=x)\)
        \item[(c)] \(\sum_{y\in\cS}p_t^*(x,y)=\bP(N(t)<\infty\mid X(0)=x)\)
    \end{enumerate}
\end{proposition}


\begin{proof}
    (b) and (c) follow from (a), therefore only (a) is proofed.

    \dhighlight{(a)}: By deriving the same iteration as \(p_t^{(+1)}(x,y)=\dots\).
    For \(c(x)>0\)
    Condition on \(\tau_0,Z_0,Z_1:\)
    \begin{align*}
        \bP(X(t)=y,N(t)<n+1\mid \tau_0=s,Z_1=z,\underbrace{Z_0}_{=X(0)}=x) &=\begin{cases}
            \delta_{x,y}& s>t\\
            \bP(X(t-s)=y,N(t-s)<n\mid X(0)=z) & s < t
        \end{cases}
    \end{align*}
    \begin{align*}
        &\bP(X(t)=y,N(t)<n+1\mid X(0)=x) \\
        &= \int_0^\infty \sum_{z\in \cS} \bP(X(t)=y,N(t)<n+1\mid \tau_0=s,Z_1=z,X(0)=x)\cdot \underbrace{\bP(\tau_0\in ds,Z_1=z\mid X(0)=x)}_{\text{density: }p(x,z)e^{-c(x)s}c(x)}\\
        &=\int_t^\infty ds \delta_{x,y}\underbrace{\cdot\sum_{z\in \cS} p(x,z)}_{=1}e^{-c(x)s}c(x)+\int_0^tds \sum_{z\in\cS}\bP(X(t-s)=y,N(t)<n\mid X(0)=z)p(x,z)c(x)e^{-c(x)s}\\
        &=\delta_{x,y}e^{-c(x)t}+\int_0^t ds e^{-c(x)\overbrace{s}^{t-s}}\sum_{z\in\cS\setminus\{x\}} \bP(X(\overbrace{t-s}^{s})=y,N(t)<n\mid X(0)=z)q(x,z)\\
        &\stackrel{s\mapsto t-s}{=}\delta_{x,y}e^{-c(x)t}+\int_0^t ds e^{-c(x)(t-s)}\sum_{q(x,z)}\bP(X(s)=y,N(s)<n\mid X(0)=z)
    \end{align*}
    \(\implies\) Same I.C. and recursion as for \(p_t^{(n)}(x,y)\). 
\end{proof}

\begin{theorem}\label{thm:1.24}
    The following are equivalent:
    \begin{enumerate}
        \item[(a)] The minimal solution \(p_t^*(x,y)\) is stochastic
        \item[(b)] \(\bP(N(t)<\infty)=1\forall t\)
        \item[(c)] \(\sum_{n\geq 0}\tau_n=\infty\as\)
        \item[(d)] \(\sum_{n\geq 0}\frac{1}{Z(n)}=\infty\as\)    
    \end{enumerate}
\end{theorem}
